\newpage

\section{\textbf{Conclusions}}

\bigskip

The study of non-viscous potential flows around different geometries, including a channel, a stationary cylinder, and a rotating cylinder, demonstrates the effectiveness of the numerical methodology in capturing fundamental flow characteristics. The comparison between analytical and numerical solutions for channel flow shows excellent agreement, confirming the accuracy of the numerical implementation for simple boundary conditions. In the case of flow around a stationary cylinder, the numerical results capture key features such as flow separation and stagnation points, though minor discrepancies are observed due to grid resolution limitations and numerical diffusion effects. The pressure distribution around the stationary cylinder aligns well with theoretical expectations, highlighting the symmetric nature of the potential flow and the absence of lift forces.

For the rotating cylinder case, the numerical solution successfully reproduces the overall circulation effects, with streamlines indicating the expected lift generation. However, discrepancies in the streamline pattern near the cylinder suggest that numerical diffusion and resolution of the vortex structure could be improved. The calculated lift and drag coefficients provide valuable insights into the aerodynamic forces acting on the cylinder, with results showing reasonable agreement with analytical predictions. 

Overall, the numerical methodology proves to be robust for modeling potential flow scenarios, providing useful approximations of analytical solutions. Further improvements, such as higher grid resolution, enhanced numerical schemes, and refined boundary condition handling, could lead to even better agreement with theoretical models, particularly for complex cases involving rotational effects.